\documentclass[twocolumn,letterpaper,10pt]{article}
\usepackage{iasted}
\usepackage{times}
\usepackage[dvips]{graphicx}
\usepackage{CJKutf8}
 
\usepackage{mathrsfs}

\usepackage{textcomp}
\usepackage{verbatim}
\usepackage{amsmath}
\usepackage{amsfonts}
\usepackage{xspace}
\usepackage{xcolor}
\usepackage{graphicx}
\usepackage{url}
\usepackage{listings}
\usepackage{balance}
\usepackage{subfig}
\usepackage{booktabs}
\usepackage{multirow}
\usepackage{rotating}
\usepackage{fancyvrb}
\usepackage{lastpage}
\usepackage{alltt}
\usepackage{etoolbox}
\usepackage{ifdraft}
\usepackage{titlesec}
\usepackage{cleveref} % After hyperref, listings
\usepackage{fancyhdr}

\usepackage{macro}
\newenvironment{CompactItemize}{\begin{itemize}}{\end{itemize}}
\def\code#1{{\texttt{#1}}}

\begin{document}

\date{}

\title{\deferu: A Lightweight Concurrent Update Method with Deferred Processing
for Linux Kernel Scalability}

\author{
Joohyun Kyong and Sung-Soo Lim\\
School of Computer Science\\
Kookmin University\\
Seoul, Korea\\
email: joohyun0115@gmail.com, sslim@kookmin.ac.kr \\
}

\maketitle

\thispagestyle{empty}

\noindent
{\bf\normalsize ABSTRACT}\newline
{
We propose a novel light weight concurrent update method, \deferu, 
to improve performance scalability for Linux kernel on many-core systems
through eliminating lock contentions for update-heavy global data structures
during process spawning and optimizing update logs. 
The proposed \deferu is implemented into Linux kernel 3.19 and evaluated 
using representative benchmark programs. 
Our evaluation reveals that the Linux kernel with \deferu shows performance
improvement by ranging from 1.2x through 2.2x on a 120 core system.

%We make the Linux fork scale to large numbers of cores by using our 
%lightweight concurrent update method, \deferu, in which updates can proceed
%concurrently.
%\deferu provides benefits over previous schemes:it is conceptually simpler and
%easier.
%Our evaluation shows performance improvements on 120 cores ranging from 1.6x to
%2.2x for an implementation of this design in the Linux 3.19 kernel.
%In this paper, we present \deferu's design, efficient techniques for the Linux
%fork scalability, and \deferu's performance evaluation.
} \vspace{2ex}
   
\noindent
{\bf\normalsize KEY WORDS}\newline
{Scalability, Operating System, Linux, Concurrent Update}

\input{intro}
\input{related}
\input{deferu}
\input{linux}
\input{impl}
\input{eval}
\input{concl}

{
\bibliographystyle{plain}
\bibliography{ref}
}

%%%%%%%%%%%%%%%%%%%%%%%%%%%%%%%%%%%%%%%%%%%%%%%%%%%%%%%%%%%%%%%%%%%%%%%%%%%%%%%%%%%%%%%%%%%%%%%%%%%% 
\end{document}
